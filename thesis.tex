\documentclass{article}

\title{Supplemental Content-Based Image Retrieval for Keyword Search with Relevance Feedback}
\author{Matthew T. Caldwell}
\date{June 2012}

\begin{document}
\maketitle

\begin{abstract}
Traditional keyword-based image retrieval falls short when content originators fail to provide a comprehensive set of
textual annotations for their images.  While the popularity of online photo sharing communities, such as
Flickr, has resulted in the availability of a massive amount of online media, it has been shown that most users do not bother
with properly annotating that media \cite{whywetag}.  Thus, a large portion of this vibrant community of online images
may never be
utilized, as the images can not be discovered using traditional keyword-based searches.  In this paper, we introduce a
hybrid keyword- and content-based image retrieval algorithm which utilizes relevance feedback to help overcome these
limitations.  Furthermore, we demonstrate its use in a new software system that we have developed to help users
discover new, exciting images in their community and from all around the world.  We conclude by providing measures of the
efficacy of our method
and describe ways by which it may be improved.
\end{abstract}

\section{Introduction}


\begin{thebibliography}{99}
  \bibitem{whywetag} Ames, M., Naaman, M.  Why We Tag: Motivations for Annotation in Mobile and Online Media.  In
    {\em CHI 2007, April 28-May 3, 2007}.
\end{thebibliography}

\end{document}
